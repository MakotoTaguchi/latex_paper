\chapter{論文内の参照}
\label{chp:reference}

\section{脚注}
\label{sec:reference_ftnote}
	文章中に脚注を使用したい場合はfootnoteコマンドを使用します.「\verb|\footnote[番号]{内容}|」のように記述します\footnote[99]{脚注表示例.}.番号を省略した場合には自動で番号が振られます\footnote{番号省略例}.

\section{引用}
\label{sec:reference_quote}
	文献等の引用を行う場合には,quote環境またはquotation環境を使用します.\\
	|quote表示例|
	\begin{quote}
		quoteは短文の引用に用いられます.

		quoteは段落の先頭字下げを行いません.
	\end{quote}
	|quotation表示例|
	\begin{quotation}
		quotationは長文の引用に用いられます.quotationは段落の先頭字下げを行います.

		quote環境とquotation環境では環境外の文章から一段下がって表示されています.
	\end{quotation}

\section{参考文献}
\label{sec:reference_bib}
	参考文献があることを示したい場合はciteコマンドを使用します.「\verb|\cite{参考文献}|」のように記述します\cite{Webデザイン}.

	参考文献一覧は巻末に記載しています.citeコマンドの引数にはthebibliography環境内のbibitemコマンドに記載した名前を使用することで,番号が表示されます.thebibliography環境の引数は参考文献の最大数で,配列を宣言する場合と同じ考えです.特に理由がなければ99のままでよいでしょう.

\section{章節番号・図表番号}
\label{sec:reference_chapter}
	章節や図表の番号を参照する場合にはlabelコマンドとrefコマンドを使用します.事前に章節等を記述している部分にlabelコマンドで名前をつけ,呼び出す際にその名前をrefコマンドで使用することで番号を表示することができます.

	呼び出すことのできるのは番号のみのため,前後に言葉を補う必要があります.\\
	|記述例|
	\begin{verbatim}
		\chapter{論文内の参照}
		\label{chp:reference}

		この部分については第\ref{chp:reference}章を参照してください.
	\end{verbatim}
	|表示例|\
		この部分については第\ref{chp:tex_basic}章(第2章の参照名を確認)参照してください.
	